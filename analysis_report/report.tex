% Options for packages loaded elsewhere
\PassOptionsToPackage{unicode}{hyperref}
\PassOptionsToPackage{hyphens}{url}
%
\documentclass[
]{article}
\usepackage{amsmath,amssymb}
\usepackage{iftex}
\ifPDFTeX
  \usepackage[T1]{fontenc}
  \usepackage[utf8]{inputenc}
  \usepackage{textcomp} % provide euro and other symbols
\else % if luatex or xetex
  \usepackage{unicode-math} % this also loads fontspec
  \defaultfontfeatures{Scale=MatchLowercase}
  \defaultfontfeatures[\rmfamily]{Ligatures=TeX,Scale=1}
\fi
\usepackage{lmodern}
\ifPDFTeX\else
  % xetex/luatex font selection
\fi
% Use upquote if available, for straight quotes in verbatim environments
\IfFileExists{upquote.sty}{\usepackage{upquote}}{}
\IfFileExists{microtype.sty}{% use microtype if available
  \usepackage[]{microtype}
  \UseMicrotypeSet[protrusion]{basicmath} % disable protrusion for tt fonts
}{}
\makeatletter
\@ifundefined{KOMAClassName}{% if non-KOMA class
  \IfFileExists{parskip.sty}{%
    \usepackage{parskip}
  }{% else
    \setlength{\parindent}{0pt}
    \setlength{\parskip}{6pt plus 2pt minus 1pt}}
}{% if KOMA class
  \KOMAoptions{parskip=half}}
\makeatother
\usepackage{xcolor}
\usepackage[margin=1in]{geometry}
\usepackage{color}
\usepackage{fancyvrb}
\newcommand{\VerbBar}{|}
\newcommand{\VERB}{\Verb[commandchars=\\\{\}]}
\DefineVerbatimEnvironment{Highlighting}{Verbatim}{commandchars=\\\{\}}
% Add ',fontsize=\small' for more characters per line
\usepackage{framed}
\definecolor{shadecolor}{RGB}{248,248,248}
\newenvironment{Shaded}{\begin{snugshade}}{\end{snugshade}}
\newcommand{\AlertTok}[1]{\textcolor[rgb]{0.94,0.16,0.16}{#1}}
\newcommand{\AnnotationTok}[1]{\textcolor[rgb]{0.56,0.35,0.01}{\textbf{\textit{#1}}}}
\newcommand{\AttributeTok}[1]{\textcolor[rgb]{0.13,0.29,0.53}{#1}}
\newcommand{\BaseNTok}[1]{\textcolor[rgb]{0.00,0.00,0.81}{#1}}
\newcommand{\BuiltInTok}[1]{#1}
\newcommand{\CharTok}[1]{\textcolor[rgb]{0.31,0.60,0.02}{#1}}
\newcommand{\CommentTok}[1]{\textcolor[rgb]{0.56,0.35,0.01}{\textit{#1}}}
\newcommand{\CommentVarTok}[1]{\textcolor[rgb]{0.56,0.35,0.01}{\textbf{\textit{#1}}}}
\newcommand{\ConstantTok}[1]{\textcolor[rgb]{0.56,0.35,0.01}{#1}}
\newcommand{\ControlFlowTok}[1]{\textcolor[rgb]{0.13,0.29,0.53}{\textbf{#1}}}
\newcommand{\DataTypeTok}[1]{\textcolor[rgb]{0.13,0.29,0.53}{#1}}
\newcommand{\DecValTok}[1]{\textcolor[rgb]{0.00,0.00,0.81}{#1}}
\newcommand{\DocumentationTok}[1]{\textcolor[rgb]{0.56,0.35,0.01}{\textbf{\textit{#1}}}}
\newcommand{\ErrorTok}[1]{\textcolor[rgb]{0.64,0.00,0.00}{\textbf{#1}}}
\newcommand{\ExtensionTok}[1]{#1}
\newcommand{\FloatTok}[1]{\textcolor[rgb]{0.00,0.00,0.81}{#1}}
\newcommand{\FunctionTok}[1]{\textcolor[rgb]{0.13,0.29,0.53}{\textbf{#1}}}
\newcommand{\ImportTok}[1]{#1}
\newcommand{\InformationTok}[1]{\textcolor[rgb]{0.56,0.35,0.01}{\textbf{\textit{#1}}}}
\newcommand{\KeywordTok}[1]{\textcolor[rgb]{0.13,0.29,0.53}{\textbf{#1}}}
\newcommand{\NormalTok}[1]{#1}
\newcommand{\OperatorTok}[1]{\textcolor[rgb]{0.81,0.36,0.00}{\textbf{#1}}}
\newcommand{\OtherTok}[1]{\textcolor[rgb]{0.56,0.35,0.01}{#1}}
\newcommand{\PreprocessorTok}[1]{\textcolor[rgb]{0.56,0.35,0.01}{\textit{#1}}}
\newcommand{\RegionMarkerTok}[1]{#1}
\newcommand{\SpecialCharTok}[1]{\textcolor[rgb]{0.81,0.36,0.00}{\textbf{#1}}}
\newcommand{\SpecialStringTok}[1]{\textcolor[rgb]{0.31,0.60,0.02}{#1}}
\newcommand{\StringTok}[1]{\textcolor[rgb]{0.31,0.60,0.02}{#1}}
\newcommand{\VariableTok}[1]{\textcolor[rgb]{0.00,0.00,0.00}{#1}}
\newcommand{\VerbatimStringTok}[1]{\textcolor[rgb]{0.31,0.60,0.02}{#1}}
\newcommand{\WarningTok}[1]{\textcolor[rgb]{0.56,0.35,0.01}{\textbf{\textit{#1}}}}
\usepackage{graphicx}
\makeatletter
\def\maxwidth{\ifdim\Gin@nat@width>\linewidth\linewidth\else\Gin@nat@width\fi}
\def\maxheight{\ifdim\Gin@nat@height>\textheight\textheight\else\Gin@nat@height\fi}
\makeatother
% Scale images if necessary, so that they will not overflow the page
% margins by default, and it is still possible to overwrite the defaults
% using explicit options in \includegraphics[width, height, ...]{}
\setkeys{Gin}{width=\maxwidth,height=\maxheight,keepaspectratio}
% Set default figure placement to htbp
\makeatletter
\def\fps@figure{htbp}
\makeatother
\setlength{\emergencystretch}{3em} % prevent overfull lines
\providecommand{\tightlist}{%
  \setlength{\itemsep}{0pt}\setlength{\parskip}{0pt}}
\setcounter{secnumdepth}{-\maxdimen} % remove section numbering
\usepackage{booktabs}
\usepackage{longtable}
\usepackage{array}
\usepackage{multirow}
\usepackage{wrapfig}
\usepackage{float}
\usepackage{colortbl}
\usepackage{pdflscape}
\usepackage{tabu}
\usepackage{threeparttable}
\usepackage{threeparttablex}
\usepackage[normalem]{ulem}
\usepackage{makecell}
\usepackage{xcolor}
\ifLuaTeX
  \usepackage{selnolig}  % disable illegal ligatures
\fi
\IfFileExists{bookmark.sty}{\usepackage{bookmark}}{\usepackage{hyperref}}
\IfFileExists{xurl.sty}{\usepackage{xurl}}{} % add URL line breaks if available
\urlstyle{same}
\hypersetup{
  pdftitle={JSC370 Final Project},
  pdfauthor={John Chen},
  hidelinks,
  pdfcreator={LaTeX via pandoc}}

\title{JSC370 Final Project}
\author{John Chen}
\date{}

\begin{document}
\maketitle

\hypertarget{introduction}{%
\section{Introduction}\label{introduction}}

The purpose of this analysis is to attempt to create a model that best
predicts the existence of a heart disease given the body condition of an
individual. The model will also be used to examine the relationship and
significance between selected features and the response variable. The
dataset we will be using is the
\href{https://www.kaggle.com/datasets/fedesoriano/heart-failure-prediction/data}{Heart
Failure Prediction Dataset}. It contains 918 observations 11 features:
Age, Sex, Chest Pain Type, Resting blood pressure, Cholesterol, Fasting
blood sugar, Resting electrocardiogram results, maximum heart rate
achieved, exercise-induced angina, oldpeak, and the slope of the peak
exercise ST segment. Duplicates were removed.

\hypertarget{method}{%
\section{Method}\label{method}}

Our dataset is cross-sectional and combined from 5 datasets used for
heart disease research:
\href{https://archive.ics.uci.edu/dataset/45/heart+disease}{Cleveland,
Hungarian, Switzerland, Long Beach},
\href{https://archive.ics.uci.edu/dataset/145/statlog+heart}{Stalog Data
set}. They are all from the UCI Machine Learning Repository. The dataset
will be split into 80-20 portions for training and testing respectively.
We will compare the accuracy between logistic regression, random forest,
and Extreme Gradient Boosting and interpret the best model. The
assumptions of logistic regression is met: observations are independent
from each other(each patient only has one observation), the response
variable is binary, and we will be using the logit function as the link
function. For testing, the predicted probability will be evaluated to 1
if its above 0.5, otherwise it will be evaluated to 0. For random forest
and Extreme Gradient Boosting, we will fine tune the parameters for best
accuracy.

\hypertarget{eda}{%
\subsubsection{EDA}\label{eda}}

Through basic data wrangling, this data contains no missing values and
all variables types are in their expected type, categorical predictors
will be transformed into factors for decision tree models training, and
all unique values of categorical variables make sense, and the response
variable only has two values for heart disease indication.

The distribution of heart disease is about uniform. The clustering of
the scatterplots between predictors does not show non-linear pattern.
This suggest that the regression should be a good fit.

There are 172 observations with 0 serum cholesterol which doesn't make
any sense. Based on this
\href{https://www.ncbi.nlm.nih.gov/pmc/articles/PMC6024687/}{study},
there is no evidence that there is a correlation between cholesterol and
heart disease. Therefore, it is expected that replacing 0 cholesterol
with the median will not have significant impact on training results.

\hypertarget{results}{%
\section{Results}\label{results}}

AUC of the logistic regression model is 0.89 which indicates that the
model is good at ranking the probabilities of the presence of heart
disease in the training set. Accuracy of the model is around 16\% which
indicates the it is not good at predicting new dataset. This could mean
that the model is overfit. The coefficients of the model is presented
below.

\begin{Shaded}
\begin{Highlighting}[]
\NormalTok{logistic\_table }\OtherTok{\textless{}{-}} \FunctionTok{tidy}\NormalTok{(reduced\_model)}
\FunctionTok{kable}\NormalTok{(logistic\_table, }\AttributeTok{format =} \StringTok{"latex"}\NormalTok{) }\SpecialCharTok{\%\textgreater{}\%}
  \FunctionTok{kable\_styling}\NormalTok{()}
\end{Highlighting}
\end{Shaded}

\begin{table}
\centering
\begin{tabular}{l|r|r|r|r}
\hline
term & estimate & std.error & statistic & p.value\\
\hline
(Intercept) & 0.2800439 & 0.9096977 & 0.3078428 & 0.7582020\\
\hline
SexM & 1.5169741 & 0.3116394 & 4.8677225 & 0.0000011\\
\hline
ChestPainTypeATA & -2.1393969 & 0.3907402 & -5.4752413 & 0.0000000\\
\hline
ChestPainTypeNAP & -1.5470520 & 0.2920778 & -5.2967126 & 0.0000001\\
\hline
ChestPainTypeTA & -1.7367296 & 0.5011811 & -3.4652737 & 0.0005297\\
\hline
FastingBS & 1.4607001 & 0.3012573 & 4.8486799 & 0.0000012\\
\hline
MaxHR & -0.0122388 & 0.0051253 & -2.3879102 & 0.0169445\\
\hline
ExerciseAnginaY & 0.9220335 & 0.2751079 & 3.3515347 & 0.0008036\\
\hline
Oldpeak & 0.3480843 & 0.1289783 & 2.6987824 & 0.0069594\\
\hline
ST\_SlopeFlat & 1.4988307 & 0.4682215 & 3.2011145 & 0.0013690\\
\hline
ST\_SlopeUp & -1.0444413 & 0.4871879 & -2.1438165 & 0.0320476\\
\hline
\end{tabular}
\end{table}

We can interpret the coefficient as odds ratio. For example, the odds of
male having heart disease is 0.212248 times higher than the odds of
female having heart disease.

We have the following accuracy percentage after training random forest
and extreme gradient boosting models.

\begin{Shaded}
\begin{Highlighting}[]
\NormalTok{accuracy\_table }\OtherTok{\textless{}{-}} \FunctionTok{data.frame}\NormalTok{(}
  \StringTok{\textasciigrave{}}\AttributeTok{Logistic Regression}\StringTok{\textasciigrave{}} \OtherTok{=}\NormalTok{ logistic\_por}\SpecialCharTok{*}\DecValTok{100}\NormalTok{,}
  \StringTok{\textasciigrave{}}\AttributeTok{Random Forest Model}\StringTok{\textasciigrave{}} \OtherTok{=}\NormalTok{ rf\_por}\SpecialCharTok{*}\DecValTok{100}\NormalTok{,}
  \StringTok{\textasciigrave{}}\AttributeTok{Extreme Gradient Boosting}\StringTok{\textasciigrave{}} \OtherTok{=}\NormalTok{ xgb\_por}\SpecialCharTok{*}\DecValTok{100}
\NormalTok{)}

\FunctionTok{kable}\NormalTok{(accuracy\_table, }\AttributeTok{format =} \StringTok{"latex"}\NormalTok{) }\SpecialCharTok{\%\textgreater{}\%}
  \FunctionTok{kable\_styling}\NormalTok{()}
\end{Highlighting}
\end{Shaded}

\begin{table}
\centering
\begin{tabular}{r|r|r}
\hline
Logistic.Regression & Random.Forest.Model & Extreme.Gradient.Boosting\\
\hline
16.30435 & 16.30435 & 12.5\\
\hline
\end{tabular}
\end{table}

The accuracy for all model is low. This could a sign of overfitting for
all models. This result might also indicate that the features are not
good predictors of heart disease. Refering to the ranking table of on
home webpage, slope of the peak exercise ST segment and max heart rate
seems to be extremely relevant because it was showed as the most
relevant variables in both models. It is unexpected that Cholesterol is
relatively high on the ranking since literature concluded that there is
no evidence between high cholesterol and higher risk of heart disease.
This could be because that the relationship between cholesterol and
heart disease depends on cofounders.

\hypertarget{conclusion}{%
\section{Conclusion}\label{conclusion}}

We have failed to train a model that accurately predict the presence of
heart disease. There is an indication of overfitting. Models like deep
learning with heavy regularization might be more fitting for this
analysis. However, we did concluded that slope of the peak exercise ST
segment and max heart rate are extremely relevant in prediction, more
specifically, Upsloping ST segment and low max max heart rate.

\end{document}
